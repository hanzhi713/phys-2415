\documentclass[../homework.tex]{subfiles}

\pagestyle{fancy}
\fancyhf{}
\rhead{Hanzhi Zhou}
\lhead{Physics 2415 Homework}
\cfoot{\thepage}

\begin{document}
\subsection{Problem 3.2}
\subsubsection*{a)}
\begin{equation*}
    C = K \varepsilon_0 \frac{A}{d}
\end{equation*}
\subsubsection*{b)}
\begin{equation*}
    r = \rho \frac{d}{A}
\end{equation*}

\subsubsection*{c)}
\begin{align*}
    V(t) &= \frac{Q(t)}{C} \\
    I(t) &= -\frac{dQ}{dt} = -\frac{V(t)}{r} = -\frac{Q(t)}{Cr}
    %  = -\frac{Q(t)}{K \varepsilon_0 \frac{A}{d} \rho \frac{d}{A}} = -\frac{Q(t)}{K \varepsilon_0 \rho}
\end{align*}

Solve the separable differential equation:
\begin{align*}
    \int \frac{1}{Q} dQ &= \int \frac{1}{Cr} dt \\
    \ln{|Q|} &= -\frac{t}{Cr} + D \\
    Q &= D e^{\frac{-t}{Cr}}
\end{align*}

Given the initial value $Q(0) = Q_0$, $D = Q_0$. 
\begin{align*}
    \frac{1}{4}Q_0 &= Q_0 e^{\frac{-t}{Cr}} \\
    \ln{\frac{1}{4}} &= \frac{-t}{Cr} \\
    t &= Cr\ln{4}
\end{align*}

Therefore $Cr\ln{4}$ amount of time is required for the charge to decrease from $Q_0$ to $\frac{1}{4}Q_0$

\subsubsection*{d)}
\begin{align*}
    dE &= I(t)^2~r~dt = \left(\frac{dQ}{dt}\right)^2~r~dt \\
     &= \left(-\frac{Q_0}{Cr} e^{\frac{-t}{Cr}}\right)^2~r~dt \\
     &= \frac{Q_0^2}{C^2 r} e^{\frac{-2t}{Cr}}~dt
\end{align*}

\subsubsection*{e)}
\begin{align*}
    E &= \frac{Q_0^2}{C^2 r} \int_{0}^{\infty} e^{\frac{-2t}{Cr}} dt \\
    &= \frac{Q_0^2}{C^2 r} \frac{C}{2r} \left(1 - e^{-\infty} \right) \\
    &= \frac{1}{2} \frac{Q_0^2}{C}
\end{align*}

\subsubsection*{f)}
\indent \indent
The energy stored in a capacitor is $\frac{1}{2}\frac{Q_0^2}{C}$ when it is fully charged to $Q_0$. It is the same as the total energy dissipated through the resistor calculated in \bf{e)}.
\end{document}