\documentclass[../homework.tex]{subfiles}

\pagestyle{fancy}
\fancyhf{}
\rhead{Hanzhi Zhou}
\lhead{Physics 2415 Homework}
\cfoot{\thepage}

\begin{document}
\subsection{Problem 3.1}
\subsubsection*{a)}
\begin{align*}
    V_{load} &= \varepsilon\frac{R_{load}}{R_{load} + r} \\
    P_{load} &= \frac{V_{load}^2}{R_{load}} = \varepsilon^2 \frac{R_{load}}{(R_{load} + r)^2}
\end{align*}
\subsubsection*{b)}
\begin{align*}
    &\frac{d}{dR}\left(\frac{R}{(R+r)^2}\right) = \frac{r-R}{(R+r)^3} = 0 \\
    &R = r
\end{align*}

Therefore the power will be largest when $R_{load} = r$
\subsubsection*{c)}
\begin{equation*}
    P_{total} = I^2 r + I^2 R = 2I^2r
\end{equation*}

Therefore 50\% of the power is dissipated by the battery 
\end{document}